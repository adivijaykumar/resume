%----------------------------------------------------------------------------------------
%	PACKAGES AND OTHER DOCUMENT CONFIGURATIONS
%----------------------------------------------------------------------------------------
\documentclass[11pt, margin, centered, letterpaper]{res}
\topmargin=-0.5in
\oddsidemargin -.5in
\evensidemargin -.5in
\textwidth=6.5in
\itemsep=0in
\parsep=0in
\newsectionwidth{1in}
\usepackage[pdftex]{graphicx}
\usepackage{etaremune}
\usepackage{enumitem}
\usepackage{wrapfig}
\usepackage{datetime}
\usepackage[dvipsnames]{xcolor}
\usepackage{helvet}
\usepackage{comment}
\usepackage{lastpage}
\usepackage[colorlinks = true,
            linkcolor = Black,
            urlcolor  = BrickRed,
            citecolor = BrickRed,
            anchorcolor = BrickRed]{hyperref}

\usepackage{fancyhdr}
\pagestyle{fancy}
\headheight 0pt
\footskip -10pt
\renewcommand{\headrulewidth}{0.0pt}
% \renewcommand{\footrulewidth}{0pt}
% \renewcommand\footrule{\moveleft\hoffset\vbox{\hrule width 19cm height 0.5pt}}
\cfoot{ \textcolor{black}{\thepage \ of \pageref{LastPage}}}

\setlength{\textwidth}{6.5in} % Text width of the document
\setlength{\textheight}{710pt}
\makeatletter %% <- change @ so that it can be used in macro names
  %% Define the resume key for etaremune:
  \define@boolkey[EM]{etaremune}{resume}[true]{}
  \presetkeys[EM]{etaremune}{resume=false}{} %% <- false by default

  %% Increase starting value of previous etaremune environment if resuming
  \AtEndEnvironment{etaremune}{%
    \ifEM@etaremune@resume
      \EM@resumewrite{\csname EM@prevlist@\@roman{\@enumdepth}\endcsname}{\EM@currlist}%
    \fi
    \expandafter\xdef\csname EM@prevlist@\@roman{\@enumdepth}\endcsname{\EM@currlist}%
  }
  \newcommand*\EM@resumewrite[2]{% %% Expand arguments and then call EM@resume@write@
    \begingroup
      \edef\temp{\noexpand\EM@resumewrite@
        {\expandafter\string\csname etaremune@#1\endcsname}%
        {\expandafter\string\csname etaremune@#2\endcsname}}%
      \temp
    \endgroup
  }
  \newcommand*\EM@resumewrite@[2]{% %% Write to aux file
    \immediate\write\@auxout{\xdef#1{\string\noexpand\string\the\numexpr#1+\string\noexpand#2 }}%
  }
\makeatother %% <- change @ back



\begin{document}

%----------------------------------------------------------------------------------------
%	NAME AND ADDRESS SECTION
%----------------------------------------------------------------------------------------\
\begin{center}
    \hspace{-\hoffset}
    \huge {\textcolor{black}{\textbf{Aditya Vijaykumar}}}
\end{center}
\vspace{-5mm}
\moveleft\hoffset\vbox{\hrule width 19cm height 0.5pt}
\vspace{-8mm}
\begin{center}
    \hspace{-\hoffset}
    \href{mailto:aditya@utoronto.ca}{aditya@utoronto.ca} ~\textbullet~ \href{http://adivijaykumar.github.io}{Website} ~\textbullet~ \ Canadian Institute for Theoretical Astrophysics (CITA)
\end{center}
\vspace{-7mm}
\begin{resume}

\section{Reserach Interests}
Gravitational Wave Astronomy and Astrophysics, Tests of General Relativity and Cosmology, Scientific Computing 
\\

%----------------------------------------------------------------------------------------
%	EXPERIENCE SECTION
%----------------------------------------------------------------------------------------
\section{Employment}
\textbf{CITA Postdoctoral Fellow}
\\
\textbf{\href{https://www.icts.res.in/}{Canadian Institute for Theoretical Astrophysics (CITA), Toronto}}\\
\emph{Independent research fellowship} \hfill Sep 2023 - \textit{Present}\\
\textit{Member of the LIGO Scientific Collaboration}

\textbf{Graduate Student}
\\
\textbf{\href{https://www.icts.res.in/}{International Centre for Theoretical Sciences (ICTS-TIFR), Bengaluru}}\\
\emph{Mentored by \href{https://home.icts.res.in/~ajith/Home.html}{Prof. Parameswaran Ajith}} \hfill Aug 2018 - Aug 2023\\
\textit{Member of the LIGO Scientific Collaboration and the LIGO-India Scientific Collaboration}

\textbf{Fulbright-Nehru Doctoral Research Fellow}
\\
\textbf{\href{https://www.icts.res.in/}{Department of Physics, The University of Chicago}}\\
\emph{Mentored by \href{https://home.icts.res.in/~ajith/Home.html}{Prof. Daniel Holz}} \hfill Aug 2022 - Mar 2023


\begin{comment}
%\section{Research Experience}

%\textbf{Visiting Student (Masters Thesis)}
%\\
%\textbf{\href{http://chep.iisc.ac.in/}{Centre for High Energy Physics (CHEP), Indian Institute of Science (IISc), Bengaluru, India}}\\
%\emph{Mentored by \href{http://chep.iisc.ac.in/Personnel/pages/chethan/index.html}{Prof. Chethan Krishnan}} \hfill July 2017 - April 2018\\
%\textbf{Complexity in context of Locality, Entanglement and Quantum Gravity} - We aim to extract lessons for quantum gravity by studying the interplay of entanglement and locality in a few physical systems. We reviewed the various conjectures on complexity and related concepts, and attempted calculating complexity for different field theories.

%\textbf{Research Project}\\
%\textbf{\href{http://www.bits-pilani.ac.in/}{Birla Institute of Technology and Science (BITS), Pilani}}\\
%\emph{Mentored by \href{http://www.bits-pilani.ac.in/Pilani/jayendra/Profile}{Prof. J N Bandyopadhyay} and \href{http://www.bits-pilani.ac.in/Pilani/tapomoy/Profile}{Prof. Tapomoy G Sarkar}} \hfill Aug 2016 - June 2017\\
%\textbf{Entanglement Production in Coupled Chaotic Systems} - A computational study of chaotic properties of a coupled chaotic system. We considered a coupled top, and using some approximation methods to the Hamiltonian, found the chaotic properties within some parameter ranges. A statistical analysis of the properties followed, with results.

%\textbf{Summer Research Intern}
%\\
%\textbf{\href{http://www.iucaa.ernet.in/}{The Inter-University Centre for Astronomy and Astrophysics (IUCAA), Pune, India}}
%\\
%\emph{Mentored by \href{http://www.iucaa.ernet.in/~anand/}{Prof. Raghunathan Srianand}} \hfill May 2016 - July 2016\\
%\textbf{Analysis of Quasar Absorption Lines from SDSS Photometric Data} - Using photometric data of quasars with absorbers in their line of sight taken from the Sloan Digital Sky Survey (SDSS), we used some image processing techniques such as stacking to establish a correspondence between the results already obtained from the spectral data also taken from SDSS. We used some statistical methods to establish this result. \\
%\\
%\textbf{Summer Research Intern}\\
%\textbf{\href{http://www.ncra.tifr.res.in/}{The National Centre for Radio Astrophysics (NCRA-TIFR), Pune, India}}\\
%\emph{Mentored by \href{http://www.ncra.tifr.res.in/ncra/people/academic/ncra-faculty/Yashwant_Gupta}{Prof. Yashwant Gupta}} \hfill May 2015 - July 2015\\
%\textbf{Testing and Debugging the Transient Detection Pipeline of GMRT} - Squashed crucial bugs and tested the transient pipeline using test data from known and reliable transient sources such as pulsars. Also reviewed key concepts of radio astronomy and pulsar astrophysics in the process.


%----------------------------------------------------------------------------------------
%----------------------------------------------------------------------------------------
%	RELEVANT COURSE SECTION
%----------------------------------------------------------------------------------------

\section{Projects}
\begin{itemize}[leftmargin=*]
	\item \textbf{Aditya Vijaykumar}, MV Saketh, Sumit Kumar, Parameswaran Ajith, Tirthankar Roy Choudhury.\\
	\textit{Probing the cosmological large-scale structure using gravitational-wave observations} \\
	\textit{(manuscript under LIGO PnP review, to be submitted to arXiv soon)}
	\\
	\item \textbf{Aditya Vijaykumar}, Shasvath Kapadia, Parameswaran Ajith.\\
	\textit{Constraining the time-variation of the Gravitational constant using gravitational-wave observations of binary neutron stars} \\
	\textit{(manuscript under LIGO PnP review, to be submitted to arXiv soon)}
	\\
	\item \textbf{Aditya Vijaykumar}, Nathan Johnson-McDaniel, Rahul Kashyap, Arunava Mukherjee, Parameswaran Ajith.\\	
	\textit{Constraints on Black Hole Mimickers from the Gravitational-wave Transient Catalog (GWTC) -1 }
	\\
	\item Apratim Ganguly, \textbf{Aditya Vijaykumar}, Abhirup Ghosh, Parameswaran Ajith.\\	
	\textit{Probing General Relativity from the consistency of inspiral and merger-ringdown of Binary Black Holes}


\end{itemize}

\end{comment}

%----------------------------------------------------------------------------------------
%	EDUCATION SECTION
%----------------------------------------------------------------------------------------
\section{Education}
\textbf{\href{https://www.icts.res.in/}{International Centre for Theoretical Sciences (ICTS-TIFR), Bengaluru}}\\
\textbf{Research Scholar and Graduate Student in Physics} \hfill 2018 - 2023

\textbf{\href{http://www.bits-pilani.ac.in/}{Birla Institute of Technology and Science (BITS), Pilani}}\\
\textbf{M.Sc. (Hons.) Physics and B.E. (Hons.) Mechanical Engineering} \hfill 2013 - 2018


\section{References}
\begin{tabular}{lr}
% Referee 1
\begin{minipage}[t]{3in}
Prof. Parameswaran Ajith\\
International Centre for Theoretical Sciences (ICTS-TIFR),\\
Shivakote, Hesaraghatta-Hobli,\\
Bengaluru, 560089, India.\\
\href{mailto:ajith@icts.res.in}{ajith@icts.res.in}
\end{minipage}
&
% Referee 2
\begin{minipage}[t]{3in}
Prof. Daniel E. Holz\\
The University of Chicago,\\
Michelson Center for Physics,\\
Chicago, IL 60637, USA.\\
\href{mailto:holz@uchicago.edu}{holz@uchicago.edu}
\end{minipage}
\\
\\ % Additional newline for spacing.
\begin{minipage}[t]{3in}
Prof. Maya Fishbach\\
Canadian Institute for Theoretical Astrophysics,\\
60 St George St,\\
Toronto, ON M5S 3H8, Canada.\\
\href{mailto:fishbach@cita.utoronto.ca}{fishbach@cita.utoronto.ca}
\end{minipage}
&
% Referee 2
\begin{minipage}[t]{3in}
Prof. Bangalore S Sathyaprakash\\
Pennsylvania State University,\\
312 Whitmore Laboratory,\\
4575 Pollock Rd,\\
State College, PA 16801, USA\\
\href{mailto:bss25@psu.edu}{bss25@psu.edu}
\end{minipage}
\\
\\ % Additional newline for spacing.
\begin{minipage}[t]{3in}
Prof. Shasvath J. Kapadia\\
Inter-University Centre for Astronomy and Astrophysics,\\
Post Bag 4, Ganeshkhind,\\
Pune, 411007, India\\
\href{mailto:shasvath.kapadia@iucaa.in}{shasvath.kapadia@iucaa.in}
\end{minipage}

\end{tabular}
\section{Seminars and Talks}
\begin{itemize}[leftmargin=*]
    %item Seoul National Uni, IUCAA, IIT Gandhinagar, LVK Meeting 2023, IISER Pune Open data workshop (this whole section probably needs a lot of restructuring)
    \item \textit{Probing host environments of gravitational-wave sources} at \textbf{CITA}, January 2024 (Invited talk)
    \item \textit{Accelerating binaries and their gravitational-wave signatures} at \textbf{TASTY, Department of Astronomy and Astrophysics, University of Toronto}, January 2024 (Invited talk)
    \item \textit{Accelerating gravitational wave sources} at \textbf{Joint CITA-PI Gravitational waves meeting}, October 2023 (Contributed talk)
    \item \textit{Fast Likelihood Evaluation with Relative Binning} at\textbf{ IUCAA, Pune}, July 2023 (Invited seminar)
    \item \textit{Testing General Relativity with Gravitational Waves:\ Opportunities and Challenges} at \textbf{IIT-Gandhinagar}, June 2023 (Invited seminar)
    \item \textit{Fast Likelihood Evaluation with Relative Binning} at \textbf{Seoul National University}, October 2022 (Invited online seminar)
    \item \textit{Standard Sirens and Large Scale Structure} at \textbf{The Quest for Precision Gravitational-wave Cosmology, The University of Chicago}, September 2022 (Invited Talk)    \item \textit{Gravitational-wave probes of astrophysics and cosmology: Large Scale Clustering and Lensing} at \textbf{IGC, Pennsylvania State University}, August 2022 (Invited Seminar)
	 \item \textit{Constraints on the time variation of the gravitational constant using binary neutron star observations} at \textbf{Second Chennai Symposium on Gravitation and Cosmology}, February 2022 (Invited online Seminar)
	 \item \textit{Probing Large Scale Structure using Binary Black Hole Observations} at \textbf{\textit{Instituut-Lorentz} for Theoretical Physics, Leiden University}, June 2020 (Invited online seminar)
	 \item \textit{Constraints on Black Hole Mimickers using GWTC-1} at \textbf{ICTS In-house Symposium}, February 2020 (Contributed Poster)
	 \item \textit{Probing Large Scale Structure using Binary Black Hole Observations} at \textbf{ICTS In-house Symposium}, ICTS, Bengaluru, India, February 2020 (Contributed Talk)
	 \item \textit{Probing Large Scale Structure using Binary Black Hole Observations} at \textbf{International Conference on Gravitation \& Cosmology}, IISER, Mohali, India, December, 2019 (Contributed Talk)
	 \item \textit{Probing Large Scale Structure using Binary Black Hole Observations} at \textbf{The Inter-University Centre for Astronomy and Astrophysics (IUCAA)}, Pune, India, September 2019 (Invited Talk)
	 \item \textit{Probing Large Scale Structure using Binary Black Hole Observations} at \textbf{Max Planck Institute for Gravitational Physics}, Hannover, Germany, June 2019 (Invited Talk)
	\item \textit{Probing Large Scale Structure using Binary Black Hole Observations} at \textbf{GR22 and Amaldi13}, Valencia, Spain, July 2019 (Contributed Talk)
	\item \textit{Gravitational Lensing from Orbiting Binary} at the \textbf{Paper Presentation competition of APOGEE 2017}, BITS Pilani, India (\textit{Contributed Talk, First runner-up})

\end{itemize}

\section{Teaching}
\begin{itemize}[leftmargin=*]
	\item Instructor and organizer, \textbf{LIGO-Virgo Collaboration Gravitational-Wave Open Data Workshop \#5 and \#6} at ICTS.
	\item Tutor for the \textbf{Numerical Relativity} graduate course, ICTS, Jan-April 2022.
	\item Co-organizer and tutor, \textbf{ICTS Workshop on Parameter Estimation with bilby}, ICTS, Bengaluru, India, August 2020 (Online)
	\item Tutor, \textbf{Light and Beyond---Summer Course for Undergraduate Students by Prof. Rajaram Nityananda}, June 2020 (Online)
	\item Tutor, \textbf{LIGO-Virgo Collaboration Gravitational-Wave Open Data Workshop \#3}, May 2020 (Online)
	\item Tutor for the following mini-courses, \textbf{ICTS Summer Schools on Gravitational Wave Astronomy}, ICTS, Bengaluru, India:
	\begin{enumerate}
            \item \textit{Compact binary evolution, rates and population modelling}, June 2022.
		\item \textit{Astrophysical Stochastic GW Foreground}, July 2021.
		\item \textit{Numerical Hydrodynamics}, May 2020.
		\item \textit{Advanced General Relativity}, July 2019.
	\end{enumerate}
\end{itemize}


\section{Mentorship}
\begin{itemize}[leftmargin=*]
    \item Kaustubh Gupta (IISER, Pune) \hfill May 2022 - \textit{Present}
    \item Adhrit Ravichandran (IIT Roorkee $\rightarrow$ UMass Dartmouth) \hfill Sep 2021 - Aug 2022
	\item Kruthi Krishna (IISc $ \rightarrow $ Radboud University) \hfill Sep 2020 - Aug 2021
	\item Harsh Narola (IISER, Tirupati $ \rightarrow $ Utrecht University) \hfill Sep 2020 - Aug 2021
\end{itemize}

\section{Other Conferences and Meetings}
\begin{itemize}[leftmargin=*]
	\item Semester Participant, \textbf{Advances in Computational Relativity}, ICERM, Brown University, USA. September 2020 - December 2020 (Online)
	\item	Participant, \textbf{Discussion Meeting - Astrophysics of Supermassive Black Holes}, ICTS, Bengaluru, India, December 2019
	\item Participant, \textbf{Discussion Meeting - Future of Gravitational Wave Astronomy}, ICTS, Bengaluru, India, August 2019
	\item Participant, \textbf{ICTS Summer School on Gravitational Wave Astronomy}, ICTS, Bengaluru, India, July 2017, July 2018, July 2019, May 2020, July 2021, May 2022.

\end{itemize}
\section{Outreach}
\begin{itemize}[leftmargin=*]
	\item Co-PI of the \textit{IndiaBioscience Outreach Grant} to communicate science using stage theatre.
	\item Panelist at the \textit{Bengaluru: The Astronomy City}, a Q\&A event organized for \textbf{National Science Day}, February 2022.
	\item Mediator for the \textbf{\href{https://bengaluru.sciencegallery.com/contagion-archive}{Contagion} Exhibition}, Science Gallery Bengaluru, April-July 2021.
	\item Moderated a discussion with Prof. Smitha Vishveshwara on her collaborative science theatre project \textit{Quantum Voyages} as a part of \textbf{\href{https://cosmic-zoom.in/}{Cosmic Zoom} Online Exhibition}, April 2021
	\item Articles on the \textbf{ICTS blog}:
	\begin{enumerate}
		\item \href{https://blog.icts.res.in/blog/conversation-icts-scientists-studying-indian-monsoon}{A Conversation with ICTS Scientists Studying the Indian Monsoon}, November 2019
		\item \href{https://blog.icts.res.in/blog/summer-school-gravitational-wave-astronomy}{Summer School on Gravitational Wave Astronomy}, November 2019
	\end{enumerate}	
	\item Talk titled \textit{The Whats, Whys and Hows of Gravitational-wave Astronomy}, \textbf{BMS College of Engineering, Bengaluru}, November 2019
	\item Talk titled \textit{Gravitational Waves - A New Tool for Cosmology!} at \textbf{Vigyan Samagam}, Visvesvaraya Industrial and Technological Museum, Bengaluru, India, August 2019

\end{itemize}




%----------------------------------------------------------------------------------------
%	TECHNICAL SKILLS SECTION
%----------------------------------------------------------------------------------------
\section{Technical \hspace{2mm} Skills}
\textbf{Programming Languages} - Python, C, C++, Shell Script\\
\textbf{Softwares} - MATLAB, Mathematica \\
\textbf{Tools/Frameworks} - \LaTeX, Git



\section{Scores and Awards}
\begin{itemize}[leftmargin=*]
 \item \href{https://www.usief.org.in/Fulbright-Nehru-Doctoral-Research-Fellowships.aspx}{Fulbright-Nehru Doctoral Research Fellowship} 2023 (Host Institution: The University of Chicago)
 \item ICTS Graduate Fellowship 2018-2023
 \item Secured all-India rank 21 in the \href{https://www.jest.org.in/}{Joint Entrance Screening Test (JEST)}, 2018 for admission into Physics PhD programmes in India
 \item Awarded the \href{https://www.icts.res.in/academic/summer-research-program}{ICTS S.N. Bhatt Memorial Excellence Fellowship}, 2018
 \item Scored 960/990 on the \href{https://www.ets.org/gre/subject/about/content/physics}{Subject GRE in Physics}, October 2017
 \item Selected for the \href{http://web-japps.ias.ac.in:8080/fellowship2018/}{Summer Research Fellowship} of the Indian Academy of Sciences in 2016
 \item Recepient of the \href{http://www.inspire-dst.gov.in/scholarship.html}{INSPIRE-DST Scholarship for Higher Education} for the period 2013 to 2018
\end{itemize}

% \begin{enumerate}[leftmargin=*]
%  \item Prof. Parameswaran Ajith, ICTS -- \href{mailto:ajith@icts.res.in}{ajith@icts.res.in}
%   \item Prof. Daniel Holz, The University of Chicago -- \href{mailto:holz@uchicago.edu}{holz@uchicago.edu}
%   \item Prof. Maya Fishbach, CITA -- \href{mailto:fishbach@cita.utoronto.ca}{fishbach@cita.utoronto.ca}
%  \item Dr. Shasvath Kapadia, IUCAA -- \href{mailto:shasvath.kapadia@iucaa.in}{shasvath.kapadia@icts.res.in}
%  \item Prof. Bangalore S Sathyaprakash, Pennsylvania State University -- \href{mailto:bss25@psu.edu}{bss25@psu.edu}
%  \item Prof. Bala Iyer, ICTS -- \href{mailto:bala.iyer@icts.res.in}{bala.iyer@icts.res.in}
% \end{enumerate}
\end{resume}
\end{document}
