%----------------------------------------------------------------------------------------
%	PACKAGES AND OTHER DOCUMENT CONFIGURATIONS
%----------------------------------------------------------------------------------------
\documentclass[margin, centered]{res}
\topmargin=-0.5in
\oddsidemargin -.5in
\evensidemargin -.5in
\textwidth=6.5in
\itemsep=0in
\parsep=0in
\newsectionwidth{1in}
\usepackage[pdftex]{graphicx}
\usepackage{enumitem}
\usepackage{wrapfig}
\usepackage{helvet}
\usepackage[colorlinks = true,
            linkcolor = blue,
            urlcolor  = blue,
            citecolor = blue,
            anchorcolor = blue]{hyperref}
\setlength{\textwidth}{6.5in} % Text width of the document
\setlength{\textheight}{720pt}

\begin{document}

%----------------------------------------------------------------------------------------
%	NAME AND ADDRESS SECTION
%----------------------------------------------------------------------------------------\
\begin{center}
    \hspace{-\hoffset}
    \huge \textbf{Aditya Vijaykumar}
\end{center}

\vspace{-5mm}
\moveleft\hoffset\vbox{\hrule width 19cm height 0.5pt}
\vspace{-8mm}
\begin{center}
    \hspace{-\hoffset}
    \href{mailto:vijaykumar.aditya@gmail.com}{vijaykumar.aditya@gmail.com} ~\textbullet~ \(+91\) 8830204638 ~\textbullet~ \ International Centre for Theoretical Sciences, Bengaluru, Karnataka, India.
\end{center}
\vspace{-7mm}
\begin{resume}

%----------------------------------------------------------------------------------------
%	EDUCATION SECTION
%----------------------------------------------------------------------------------------

\section{Education}
\textbf{\href{https://www.icts.res.in/}{International Centre for Theoretical Sciences (ICTS-TIFR), Bengaluru}}\\
\textbf{Research Scholar and Graduate Student in Physics} \hfill 2018 - \textit{Present}

\textbf{\href{http://www.bits-pilani.ac.in/}{Birla Institute of Technology and Science (BITS), Pilani}}\\
\textbf{M.Sc. (Hons.) Physics and B.E. (Hons.) Mechanical Engineering} \hfill 2013 - 2018
%\begin{itemize}
 %\item CGPA of \textbf{9.47}/10 (Dec 2015)
%\end{itemize}

\textbf{High School} - \textbf{\href{http://www.stvincentspune.com/}{St. Vincent's High School, Pune}} (Maharashtra HSC) \hfill 2011 - 2013 \\
\textbf{Secondary School} - \textbf{\href{http://www.rosaryedu.org}{Rosary High School, Pune}} (Maharashtra SSC) \hfill 1999 - 2011

%----------------------------------------------------------------------------------------
%	EXPERIENCE SECTION
%----------------------------------------------------------------------------------------
\section{Research Experience}

\textbf{Visiting Student (Masters Thesis)}
\\
\textbf{\href{http://chep.iisc.ac.in/}{Centre for High Energy Physics (CHEP), Indian Institute of Science (IISc), Bengaluru, India}}\\
\emph{Mentored by \href{http://chep.iisc.ac.in/Personnel/pages/chethan/index.html}{Prof. Chethan Krishnan}} \hfill July 2017 - April 2018\\
\textbf{Complexity in context of Locality, Entanglement and Quantum Gravity} - We aim to extract lessons for quantum gravity by studying the interplay of entanglement and locality in a few physical systems. We reviewed the various conjectures on complexity and related concepts, and attempted calculating complexity for different field theories.

\textbf{Research Project}\\
\textbf{\href{http://www.bits-pilani.ac.in/}{Birla Institute of Technology and Science (BITS), Pilani}}\\
\emph{Mentored by \href{http://www.bits-pilani.ac.in/Pilani/jayendra/Profile}{Prof. J N Bandyopadhyay} and \href{http://www.bits-pilani.ac.in/Pilani/tapomoy/Profile}{Prof. Tapomoy G Sarkar}} \hfill Aug 2016 - June 2017\\
\textbf{Entanglement Production in Coupled Chaotic Systems} - A computational study of chaotic properties of a coupled chaotic system. We considered a coupled top, and using some approximation methods to the Hamiltonian, found the chaotic properties within some parameter ranges. A statistical analysis of the properties followed, with results.

\textbf{Summer Research Intern}
\\
\textbf{\href{http://www.iucaa.ernet.in/}{The Inter-University Centre for Astronomy and Astrophysics (IUCAA), Pune, India}}
\\
\emph{Mentored by \href{http://www.iucaa.ernet.in/~anand/}{Prof. Raghunathan Srianand}} \hfill May 2016 - July 2016\\
\textbf{Analysis of Quasar Absorption Lines from SDSS Photometric Data} - Using photometric data of quasars with absorbers in their line of sight taken from the Sloan Digital Sky Survey (SDSS), we used some image processing techniques such as stacking to establish a correspondence between the results already obtained from the spectral data also taken from SDSS. We used some statistical methods to establish this result. \\
\\
\textbf{Summer Research Intern}\\
\textbf{\href{http://www.ncra.tifr.res.in/}{The National Centre for Radio Astrophysics (NCRA-TIFR), Pune, India}}\\
\emph{Mentored by \href{http://www.ncra.tifr.res.in/ncra/people/academic/ncra-faculty/Yashwant_Gupta}{Prof. Yashwant Gupta}} \hfill May 2015 - July 2015\\
\textbf{Testing and Debugging the Transient Detection Pipeline of GMRT} - Squashed crucial bugs and tested the transient pipeline using test data from known and reliable transient sources such as pulsars. Also reviewed key concepts of radio astronomy and pulsar astrophysics in the process.


%----------------------------------------------------------------------------------------
%----------------------------------------------------------------------------------------
%	RELEVANT COURSE SECTION
%----------------------------------------------------------------------------------------

\section{Conferences, Schools and Talks}
\begin{itemize}[leftmargin=*]
 \item Participant, \textbf{Discussion Meeting - Future of Gravitational Wave Astronomy}, ICTS, Bengaluru, India, August 2019
 \item Outreach talk titled \textit{Gravitational Waves - A New Tool for Cosmology!} at \textbf{Vigyan Samagam}, Visvesvaraya Industrial and Technological Museum, Bengaluru, India, August 2019
 \item Participant and Tutor for the \textit{Advanced General Relativity} mini-course, \textbf{ICTS Summer School on Gravitational Wave Astronomy}, ICTS, Bengaluru, India, July 2019
 \item Talk titled \textit{Probing Large Scale Structure using Binary Black Hole Observations} at \textbf{GR22 and Amaldi13}, Valencia, Spain, July 2019
 \item Talk titled \textit{Probing Large Scale Structure using Binary Black Hole Observations} at \textbf{Max Planck Institute for Gravitational Physics}, Hannover, Germany, June 2018
 \item Participant, \textbf{ICTS Summer School on Gravitational Wave Astronomy}, ICTS, Bengaluru, India, July 2018
 \item Participant, \textbf{ICTS Summer School on Gravitational Wave Astronomy}, ICTS, Bengaluru, India, July 2017
 \item Talk titled \textit{Gravitational Lensing from Orbiting Binary} at the \textbf{Paper Presentation competition of APOGEE 2017}, BITS Pilani, India (\textit{First runner-up})
 \end{itemize}
%----------------------------------------------------------------------------------------
%	TECHNICAL SKILLS SECTION
%----------------------------------------------------------------------------------------
\section{Technical \hspace{2mm} Skills}
\textbf{Programming Languages} - Python, C, C++, Shell Script\\
\textbf{Softwares} - MATLAB, Maple \\
\textbf{Tools/Frameworks} - \LaTeX, Git

%---------------------------------------------------------------------------------------
%	PUBLICATION SECTION
%---------------------------------------------------------------------------------------

%\section{Publications}
%\begin{itemize}[leftmargin=*]
%\item Devanshu J, \textbf{Ashish K}, Rakshit S, Sameer S, ``Recommendation Techniques for Adaptive E-learning'', Advances in Computer Science and Information Technology, vol. 2, No. 1, 2015. \href{https://drive.google.com/file/d/0B6A-3_6rwie9bS1OaFdzbW9BZXM/view?usp=sharing}{view here}
%\item \textbf{Ashish Kedia} and Anusha Prakash, "Data Synchronization on Android Clients", International Conference on Communication Software and Networks, June 6-7$^{th}$, 2015, Chengdu, China. \href{http://ieeexplore.ieee.org/xpl/articleDetails.jsp?reload=true&arnumber=7296156}{view here}
%\end{itemize}


%----------------------------------------------------------------------------------------
%	ACHIEVEMENT SECTION
%----------------------------------------------------------------------------------------

\section{Scores and Awards}
\begin{itemize}[leftmargin=*]
 \item Scored 960/990 on the \href{https://www.ets.org/gre/subject/about/content/physics}{Subject GRE in Physics}, October 2017
 \item Secured all-India rank 21 in the \href{https://www.jest.org.in/}{Joint Entrance Screening Test (JEST)}, 2018 for admission into Physics PhD programmes in India
 \item Awarded the \href{https://www.icts.res.in/academic/summer-research-program}{ICTS S.N. Bhatt Memorial Excellence Fellowship}, 2018
 \item Selected for the \href{http://web-japps.ias.ac.in:8080/fellowship2018/}{Summer Research Fellowship} of the Indian Academy of Sciences in 2016
 \item Recepient of the \href{http://www.inspire-dst.gov.in/scholarship.html}{INSPIRE-DST Scholarship for Higher Education} for the period 2013 to 2018
\end{itemize}

%----------------------------------------------------------------------------------------
%	HOBBIES SECTION
%----------------------------------------------------------------------------------------

%\section{Hobbies}
%Blogging, Reading, Star Gazing, Editing Wiki Pages, Solving Puzzle

%\section{More}
%Please visit \href{https://adivijaykumar.github.io/academic/}{https://adivijaykumar.github.io/academic/}

\end{resume}
\end{document}
