%----------------------------------------------------------------------------------------
%	PACKAGES AND OTHER DOCUMENT CONFIGURATIONS
%----------------------------------------------------------------------------------------
\documentclass[margin, centered]{res}
\topmargin=-0.5in
\oddsidemargin -.5in
\evensidemargin -.5in
\textwidth=6.5in
\itemsep=0in
\parsep=0in
\newsectionwidth{1in}
\usepackage[pdftex]{graphicx}
\usepackage{enumitem}
\usepackage{wrapfig}
\usepackage{helvet}
\usepackage[colorlinks = true,
            linkcolor = blue,
            urlcolor  = blue,
            citecolor = blue,
            anchorcolor = blue]{hyperref}
\setlength{\textwidth}{6.5in} % Text width of the document
\setlength{\textheight}{720pt}

\begin{document}

%----------------------------------------------------------------------------------------
%	NAME AND ADDRESS SECTION
%----------------------------------------------------------------------------------------\
\begin{center}
    \hspace{-\hoffset}
    \huge \textbf{Aditya Vijaykumar}
\end{center}

\vspace{-5mm}
\moveleft\hoffset\vbox{\hrule width 19cm height 0.5pt}
\vspace{-8mm}
\begin{center}
    \hspace{-\hoffset}
    \href{mailto:vijaykumar.aditya@gmail.com}{vijaykumar.aditya@gmail.com} ~\textbullet~ \(+91\) 8830204638 ~\textbullet~ \ CHEP, Indian Institute of Science, Bengaluru, Karnataka, India.
\end{center}
\vspace{-7mm}
\begin{resume}

%----------------------------------------------------------------------------------------
%	EDUCATION SECTION
%----------------------------------------------------------------------------------------
\section{Research Interests}
Theoretical Physics\\


\section{Education}
\textbf{\href{http://www.bits-pilani.ac.in/}{Birla Institute of Technology and Science (BITS), Pilani}}\\
\textbf{M.Sc. (Hons.) Physics and B.E. (Hons.) Mechanical Engineering} \hfill 2013 - 2018 (Expected)
%\begin{itemize}
 %\item CGPA of \textbf{9.47}/10 (Dec 2015)
%\end{itemize}

\textbf{High School} - \textbf{\href{http://www.stvincentspune.com/}{St. Vincent's High School, Pune}} (Maharashtra HSC) - 94.27\% \hfill 2011 - 2013 \\
\textbf{Secondary School} - \textbf{\href{http://www.rosaryedu.org}{Rosary High School, Pune}} (Maharashtra SSC) - 93.27\% \hfill 1999 - 2011

%----------------------------------------------------------------------------------------
%	EXPERIENCE SECTION
%----------------------------------------------------------------------------------------
\section{Research Experience}
\textbf{Visiting Student (Masters Thesis)}
\textbf{\href{http://chep.iisc.ac.in/}{Centre for High Energy Physics (CHEP), Indian Institute of Science (IISc), Bangalore, India}}\\
\emph{Mentored by \href{http://chep.iisc.ac.in/Personnel/pages/chethan/index.html}{Prof. Chethan Krishnan}} \hfill July 2017 - Present\\
\textbf{Complexity in context of Locality, Entanglement and Quantum Gravity} - We aim to extract lessons for quantum gravity by studying the interplay of entanglement and locality in tractable physical systems. So far, we have reviewed definitions and properties of complexity of quantum states, and tried to derive results about the same.
\\
\\
\textbf{Summer Research Intern}
\\
\textbf{\href{http://www.iucaa.ernet.in/}{The Inter-University Centre for Astronomy and Astrophysics (IUCAA), Pune, India}}
\\
\emph{Mentored by \href{http://www.iucaa.ernet.in/~anand/}{Prof. Raghunathan Srianand}} \hfill May 2016 - July 2016\\
\textbf{Analysis of Quasar Absorption Lines from SDSS Photometric Data} - Using photometric data of quasars with absorbers in their line of sight taken from the Sloan Digital Sky Survey (SDSS), we used some image processing techniques such as stacking to establish a correspondence between the results already obtained from the spectral data also taken from SDSS. We used some statistical methods to establish this result. \\
\\
\textbf{Summer Research Intern}\\
\textbf{\href{http://www.ncra.tifr.res.in/}{The National Centre for Radio Astrophysics (NCRA-TIFR), Pune, India}}\\
\emph{Mentored by \href{http://www.ncra.tifr.res.in/ncra/people/academic/ncra-faculty/Yashwant_Gupta}{Prof. Yashwant Gupta}} \hfill May 2015 - July 2015\\
\textbf{Testing and Debugging the Transient Detection Pipeline of GMRT} - Squashed crucial bugs and tested the transient pipeline using test data from known and reliable transient sources such as pulsars. Also reviewed key concepts of radio astronomy and pulsar astrophysics in the process.


%----------------------------------------------------------------------------------------
%	Selected Projects Section
%----------------------------------------------------------------------------------------
\section{Selected Projects}

\textbf{Gauge Theory in Particle Physics}\\
\emph{Mentored by \href{http://universe.bits-pilani.ac.in/pilani/layek/profile}{Prof. Biswanath Layek}, BITS Pilani} \hfill Aug 2016 - Dec 2016\\
A brief introduction to gauge theory and its applications in particle physics. We started off by studying gauges, their properties, and usage, and went on to apply these concepts to electromagnetism, QED, QCD and some other cases. \\
\\
\textbf{Entanglement Production in Coupled Chaotic Systems}\\
\emph{Mentored by \href{http://www.bits-pilani.ac.in/Pilani/jayendra/Profile}{Prof. J N Bandyopadhyay} and \href{http://www.bits-pilani.ac.in/Pilani/tapomoy/Profile}{Prof. Tapomoy G Sarkar}, BITS Pilani} \hfill Aug 2016 - June 2017\\
A computational study of chaotic properties of a coupled chaotic system. We considered a coupled top, and using some approximation methods to the Hamiltonian, found the chaotic properties within some parameter ranges. A rigorous statistical analysis of the properties followed, with results (a manuscript is under preparation).

\textbf{Black Holes and Naked Singularities}\\
\emph{Mentored by \href{http://www.bits-pilani.ac.in/Pilani/tapomoy/Profile}{Prof. Tapomoy G Sarkar}, BITS Pilani} \hfill Jan 2017 - June 2017\\
An in-depth study on black holes and their various aspects. Starting with a review of black holes with different metrics, we also conducted a brief review of naked singularities and paths to quantum gravity. We also revised a fair bit of general relativity in the process.


%----------------------------------------------------------------------------------------
%	RELEVANT COURSE SECTION
%----------------------------------------------------------------------------------------
\section{Relevant \hspace{2mm} Courses}
Classical Mechanics, Electromagnetic Theory, Quantum Mechanics, Mathematical Methods in Physics, Statistical Mechanics, Computational Physics, Particle Physics, General Theory of Relativity and Cosmology, Quantum Field Theory, Introductory Astronomy and Astrophysics

\section{Conferences, Schools and Talks}
\begin{itemize}[leftmargin=*]
 \item ICTS Summer School on Gravitational Wave Astronomy, ICTS, Bengaluru, India, July 2017
 \item Presented a paper titled \textit{Gravitational Lensing from Orbiting Binary} at APOGEE 2017, BITS Pilani
 \item IISc Journal Club talk titled \textit{Teleportation Through the Wormhole}, based on \href{https://arxiv.org/abs/1707.04354}{arXiv:1707.04354}
 \end{itemize}
%----------------------------------------------------------------------------------------
%	TECHNICAL SKILLS SECTION
%----------------------------------------------------------------------------------------
\section{Technical \hspace{2mm} Skills}
\textbf{Programming Languages} - Python, C, C++, Shell Script\\
\textbf{Softwares} - MATLAB, Maple \\
\textbf{Tools/Frameworks} - \LaTeX, Git

%---------------------------------------------------------------------------------------
%	PUBLICATION SECTION
%---------------------------------------------------------------------------------------

%\section{Publications}
%\begin{itemize}[leftmargin=*]
%\item Devanshu J, \textbf{Ashish K}, Rakshit S, Sameer S, ``Recommendation Techniques for Adaptive E-learning'', Advances in Computer Science and Information Technology, vol. 2, No. 1, 2015. \href{https://drive.google.com/file/d/0B6A-3_6rwie9bS1OaFdzbW9BZXM/view?usp=sharing}{view here}
%\item \textbf{Ashish Kedia} and Anusha Prakash, "Data Synchronization on Android Clients", International Conference on Communication Software and Networks, June 6-7$^{th}$, 2015, Chengdu, China. \href{http://ieeexplore.ieee.org/xpl/articleDetails.jsp?reload=true&arnumber=7296156}{view here}
%\end{itemize}


%----------------------------------------------------------------------------------------
%	ACHIEVEMENT SECTION
%----------------------------------------------------------------------------------------

\section{Scores and Awards}
\begin{itemize}[leftmargin=*]
 \item Scored 960/990 on the Subject GRE in Physics, October 2017
 \item Recepient of the INSPIRE-DST Scholarship for Higher Education for the period 2013 to 2018
 \item Selected for the Summer Research Fellowship of the Indian Academy of Sciences in 2016
\end{itemize}

\section{Co-Curricular Achievements}
\begin{itemize}[leftmargin=*]
 \item Contributor to \href{https://www.scipy.org/}{SciPy}
 \item Captain (Head) of the \href{http://www.bits-clocktower.org/}{Clock Tower Restoration Team}, BITS Pilani for academic year 2016-17.
 \item Chief Coordinator of \href{https://bits-apogee.org/2016/}{APOGEE 2016}, the 34th edition of BITS Pilani's official annual technical festival
 \item Founding Head of the Student Academic Cell, BITS Pilani, a think-tank responsible for improving the academic environment in BITS, Pilani
 \item Founded Papyrus Trails, BITS, Pilani's Literature Festival
\end{itemize}

%----------------------------------------------------------------------------------------
%	HOBBIES SECTION
%----------------------------------------------------------------------------------------

%\section{Hobbies}
%Blogging, Reading, Star Gazing, Editing Wiki Pages, Solving Puzzle

%\section{More}
%Please visit \href{https://adivijaykumar.github.io/academic/}{https://adivijaykumar.github.io/academic/}

\end{resume}
\end{document}
