%----------------------------------------------------------------------------------------
%	PACKAGES AND OTHER DOCUMENT CONFIGURATIONS
%----------------------------------------------------------------------------------------
\documentclass[margin, centered]{res}
\topmargin=-0.5in
\oddsidemargin -.5in
\evensidemargin -.5in
\textwidth=6.5in
\itemsep=0in
\parsep=0in
\newsectionwidth{1in}
\usepackage[pdftex]{graphicx}
\usepackage{etaremune}
\usepackage{enumitem}
\usepackage{wrapfig}
\usepackage[dvipsnames]{xcolor}
\usepackage{helvet}
\usepackage{comment}

\usepackage[colorlinks = true,
            linkcolor = BrickRed,
            urlcolor  = BrickRed,
            citecolor = BrickRed,
            anchorcolor = BrickRed]{hyperref}
\setlength{\textwidth}{6.5in} % Text width of the document
\setlength{\textheight}{720pt}

\begin{document}

%----------------------------------------------------------------------------------------
%	NAME AND ADDRESS SECTION
%----------------------------------------------------------------------------------------\
\begin{center}
    \hspace{-\hoffset}
    \huge {\textcolor{black}{Aditya Vijaykumar}}
\end{center}

\vspace{-5mm}
\moveleft\hoffset\vbox{\hrule width 19cm height 0.5pt}
\vspace{-8mm}
\begin{center}
    \hspace{-\hoffset}
    \href{mailto:aditya.vijaykumar@icts.res.in}{aditya.vijaykumar@icts.res.in} ~\textbullet~ \(+91\) 8830204638 ~\textbullet~ \ International Centre for Theoretical Sciences, Bengaluru,  India.
\end{center}
\vspace{-7mm}
\begin{resume}

%----------------------------------------------------------------------------------------
%	EDUCATION SECTION
%----------------------------------------------------------------------------------------
\section{Reserach Interests}
Gravitational Wave Astronomy and Astrophysics, Tests of General Relativity and Cosmology, Scientific Computing 
\\
\section{Education}
\textbf{\href{https://www.icts.res.in/}{International Centre for Theoretical Sciences (ICTS-TIFR), Bengaluru}}\\
\textbf{Research Scholar and Graduate Student in Physics} \hfill 2018 - \textit{Present}

\textbf{\href{http://www.bits-pilani.ac.in/}{Birla Institute of Technology and Science (BITS), Pilani}}\\
\textbf{M.Sc. (Hons.) Physics and B.E. (Hons.) Mechanical Engineering} \hfill 2013 - 2018

%----------------------------------------------------------------------------------------
%	EXPERIENCE SECTION
%----------------------------------------------------------------------------------------
\section{Employment}
\textbf{Graduate Student}
\\
\textbf{\href{https://www.icts.res.in/}{International Centre for Theoretical Sciences (ICTS-TIFR), Bengaluru}}\\
\emph{Mentored by \href{https://home.icts.res.in/~ajith/Home.html}{Prof. Parameswaran Ajith}} \hfill Aug 2018 - \textit{Present}\\
\textit{Member of the LIGO Scientific Collaboration and the LIGO-India Scientific Collaboration}

\textbf{Summer Research Intern}
\\
\textbf{\href{https://www.icts.res.in/}{International Centre for Theoretical Sciences (ICTS-TIFR), Bengaluru}}\\
\emph{Mentored by \href{https://home.icts.res.in/~ajith/Home.html}{Prof. Parameswaran Ajith}} \hfill May 2018 - July 2018\\
\textit{Topic - Cosmological Large-scale Structure probes using gravitational-wave observations}

\textbf{Visiting Student (Masters Thesis)}
\\
\textbf{\href{http://chep.iisc.ac.in/}{Centre for High Energy Physics (CHEP), Indian Institute of Science (IISc), Bengaluru, India}}\\
\emph{Mentored by \href{http://chep.iisc.ac.in/Personnel/pages/chethan/index.html}{Prof. Chethan Krishnan}} \hfill July 2017 - April 2018\\
\textit{Topic - Complexity in context of Locality, Entanglement and Quantum Gravity}

\textbf{Summer Research Intern}
\\
\textbf{\href{http://www.iucaa.ernet.in/}{The Inter-University Centre for Astronomy and Astrophysics (IUCAA), Pune, India}}
\\
\emph{Mentored by \href{http://www.iucaa.ernet.in/~anand/}{Prof. Raghunathan Srianand}} \hfill May 2016 - July 2016\\
\textit{Topic - Analysis of Quasar Absorption Lines from SDSS Photometric Data}

\textbf{Summer Research Intern}\\
\textbf{\href{http://www.ncra.tifr.res.in/}{The National Centre for Radio Astrophysics (NCRA-TIFR), Pune, India}}\\
\emph{Mentored by \href{http://www.ncra.tifr.res.in/ncra/people/academic/ncra-faculty/Yashwant_Gupta}{Prof. Yashwant Gupta}} \hfill May 2015 - July 2015\\
\textit{Topic - Testing and Debugging the Transient Detection Pipeline of GMRT}

\begin{comment}
%\section{Research Experience}

%\textbf{Visiting Student (Masters Thesis)}
%\\
%\textbf{\href{http://chep.iisc.ac.in/}{Centre for High Energy Physics (CHEP), Indian Institute of Science (IISc), Bengaluru, India}}\\
%\emph{Mentored by \href{http://chep.iisc.ac.in/Personnel/pages/chethan/index.html}{Prof. Chethan Krishnan}} \hfill July 2017 - April 2018\\
%\textbf{Complexity in context of Locality, Entanglement and Quantum Gravity} - We aim to extract lessons for quantum gravity by studying the interplay of entanglement and locality in a few physical systems. We reviewed the various conjectures on complexity and related concepts, and attempted calculating complexity for different field theories.

%\textbf{Research Project}\\
%\textbf{\href{http://www.bits-pilani.ac.in/}{Birla Institute of Technology and Science (BITS), Pilani}}\\
%\emph{Mentored by \href{http://www.bits-pilani.ac.in/Pilani/jayendra/Profile}{Prof. J N Bandyopadhyay} and \href{http://www.bits-pilani.ac.in/Pilani/tapomoy/Profile}{Prof. Tapomoy G Sarkar}} \hfill Aug 2016 - June 2017\\
%\textbf{Entanglement Production in Coupled Chaotic Systems} - A computational study of chaotic properties of a coupled chaotic system. We considered a coupled top, and using some approximation methods to the Hamiltonian, found the chaotic properties within some parameter ranges. A statistical analysis of the properties followed, with results.

%\textbf{Summer Research Intern}
%\\
%\textbf{\href{http://www.iucaa.ernet.in/}{The Inter-University Centre for Astronomy and Astrophysics (IUCAA), Pune, India}}
%\\
%\emph{Mentored by \href{http://www.iucaa.ernet.in/~anand/}{Prof. Raghunathan Srianand}} \hfill May 2016 - July 2016\\
%\textbf{Analysis of Quasar Absorption Lines from SDSS Photometric Data} - Using photometric data of quasars with absorbers in their line of sight taken from the Sloan Digital Sky Survey (SDSS), we used some image processing techniques such as stacking to establish a correspondence between the results already obtained from the spectral data also taken from SDSS. We used some statistical methods to establish this result. \\
%\\
%\textbf{Summer Research Intern}\\
%\textbf{\href{http://www.ncra.tifr.res.in/}{The National Centre for Radio Astrophysics (NCRA-TIFR), Pune, India}}\\
%\emph{Mentored by \href{http://www.ncra.tifr.res.in/ncra/people/academic/ncra-faculty/Yashwant_Gupta}{Prof. Yashwant Gupta}} \hfill May 2015 - July 2015\\
%\textbf{Testing and Debugging the Transient Detection Pipeline of GMRT} - Squashed crucial bugs and tested the transient pipeline using test data from known and reliable transient sources such as pulsars. Also reviewed key concepts of radio astronomy and pulsar astrophysics in the process.


%----------------------------------------------------------------------------------------
%----------------------------------------------------------------------------------------
%	RELEVANT COURSE SECTION
%----------------------------------------------------------------------------------------

\section{Projects}
\begin{itemize}[leftmargin=*]
	\item \textbf{Aditya Vijaykumar}, MV Saketh, Sumit Kumar, Parameswaran Ajith, Tirthankar Roy Choudhury.\\
	\textit{Probing the cosmological large-scale structure using gravitational-wave observations} \\
	\textit{(manuscript under LIGO PnP review, to be submitted to arXiv soon)}
	\\
	\item \textbf{Aditya Vijaykumar}, Shasvath Kapadia, Parameswaran Ajith.\\
	\textit{Constraining the time-variation of the Gravitational constant using gravitational-wave observations of binary neutron stars} \\
	\textit{(manuscript under LIGO PnP review, to be submitted to arXiv soon)}
	\\
	\item \textbf{Aditya Vijaykumar}, Nathan Johnson-McDaniel, Rahul Kashyap, Arunava Mukherjee, Parameswaran Ajith.\\	
	\textit{Constraints on Black Hole Mimickers from the Gravitational-wave Transient Catalog (GWTC) -1 }
	\\
	\item Apratim Ganguly, \textbf{Aditya Vijaykumar}, Abhirup Ghosh, Parameswaran Ajith.\\	
	\textit{Probing General Relativity from the consistency of inspiral and merger-ringdown of Binary Black Holes}


\end{itemize}



\end{comment}

\section{Publications}
\begin{etaremune}
	\begin{comment}	
	\item 
	\textbf{Aditya Vijaykumar}, M.~V.~S.~Saketh, Sumit Kumar, Parameswaran Ajith, Tirthankar Roy Choudhury\\
	\textit{Probing the large scale structure using gravitational wave observations of binary black holes},\\
	Submitted to \textit{Physical Review Letters}, \href{https://arxiv.org/abs/1907.10121}{arXiv:2004.xxxx}. 
	\end{comment}
	\item 
	\textbf{Aditya Vijaykumar}, Shasvath J. Kapadia, Parameswaran Ajith\\
	\textit{Constraints on the time variation of the gravitational constant using gravitational wave observations of binary neutron stars},\\
	To be submitted to \textit{Physical Review Letters}, \href{https://arxiv.org/abs/2003.12832}{arXiv:2003.12832}. 
	
	\item 
	P.~Virtanen {\it et al.} (incl. \textbf{Aditya Vijaykumar} as \textit{SciPy 1.0 Contributor})\\
	\textit{SciPy 1.0--Fundamental Algorithms for Scientific Computing in Python},\\
	\href{https://www.nature.com/articles/s41592-019-0686-2}{\textit{Nat Methods} 17, 261–272 (2020)},
	\href{https://arxiv.org/abs/1907.10121}{arXiv:1907.10121}.

\end{etaremune}


\section{Conferences, Schools and Talks}
\begin{itemize}[leftmargin=*]
 \item Poster titled \textit{Constraints on Black Hole Mimickers using GWTC-1} at \textbf{ICTS In-house Symposium}, ICTS, Bengaluru, India, February 2020
 \item Talk titled \textit{Probing Large Scale Structure using Binary Black Hole Observations} at \textbf{ICTS In-house Symposium}, ICTS, Bengaluru, India, February 2020
 \item	Participant, \textbf{Discussion Meeting - Astrophysics of Supermassive Black Holes}, ICTS, Bengaluru, India, December 2019
 \item Invited outreach talk titled \textit{The Whats, Whys and Hows of Gravitational-wave Astronomy}, \textbf{BMS College of Engineering, Bengaluru}, November 2019
 \item Participant, \textbf{Discussion Meeting - Future of Gravitational Wave Astronomy}, ICTS, Bengaluru, India, August 2019
 \item Talk titled \textit{Probing Large Scale Structure using Binary Black Hole Observations} at \textbf{The Inter-University Centre for Astronomy and Astrophysics (IUCAA)}, Pune, India, September 2019
 \item Outreach talk titled \textit{Gravitational Waves - A New Tool for Cosmology!} at \textbf{Vigyan Samagam}, Visvesvaraya Industrial and Technological Museum, Bengaluru, India, August 2019
 \item Participant and Tutor for the \textit{Advanced General Relativity} mini-course, \textbf{ICTS Summer School on Gravitational Wave Astronomy}, ICTS, Bengaluru, India, July 2019
 \item Talk titled \textit{Probing Large Scale Structure using Binary Black Hole Observations} at \textbf{GR22 and Amaldi13}, Valencia, Spain, July 2019
 \item Talk titled \textit{Probing Large Scale Structure using Binary Black Hole Observations} at \textbf{Max Planck Institute for Gravitational Physics}, Hannover, Germany, June 2019
 \item Participant, \textbf{ICTS Summer School on Gravitational Wave Astronomy}, ICTS, Bengaluru, India, July 2018
 \item Participant, \textbf{ICTS Summer School on Gravitational Wave Astronomy}, ICTS, Bengaluru, India, July 2017
 \item Talk titled \textit{Gravitational Lensing from Orbiting Binary} at the \textbf{Paper Presentation competition of APOGEE 2017}, BITS Pilani, India (\textit{First runner-up})
\end{itemize}
%----------------------------------------------------------------------------------------
%	TECHNICAL SKILLS SECTION
%----------------------------------------------------------------------------------------
\section{Technical \hspace{2mm} Skills}
\textbf{Programming Languages} - Python, C, C++, Shell Script\\
\textbf{Softwares} - MATLAB, Mathematica \\
\textbf{Tools/Frameworks} - \LaTeX, Git



\section{Scores and Awards}
\begin{itemize}[leftmargin=*]
 \item Scored 960/990 on the \href{https://www.ets.org/gre/subject/about/content/physics}{Subject GRE in Physics}, October 2017
 \item Secured all-India rank 21 in the \href{https://www.jest.org.in/}{Joint Entrance Screening Test (JEST)}, 2018 for admission into Physics PhD programmes in India
 \item Awarded the \href{https://www.icts.res.in/academic/summer-research-program}{ICTS S.N. Bhatt Memorial Excellence Fellowship}, 2018
 \item Selected for the \href{http://web-japps.ias.ac.in:8080/fellowship2018/}{Summer Research Fellowship} of the Indian Academy of Sciences in 2016
 \item Recepient of the \href{http://www.inspire-dst.gov.in/scholarship.html}{INSPIRE-DST Scholarship for Higher Education} for the period 2013 to 2018
\end{itemize}


\end{resume}
\end{document}
